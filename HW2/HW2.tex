% !TEX TS-program = pdflatex
% !TEX encoding = UTF-8 Unicode

% This is a simple template for a LaTeX document using the "article" class.
% See "book", "report", "letter" for other types of document.

\documentclass[12pt]{article} % use larger type; default would be 10pt

\usepackage[utf8]{inputenc} % set input encoding (not needed with XeLaTeX)

%%% Examples of Article customizations
% These packages are optional, depending whether you want the features they provide.
% See the LaTeX Companion or other references for full information.

%%% PAGE DIMENSIONS
\usepackage{geometry} % to change the page dimensions
\geometry{a4paper} % or letterpaper (US) or a5paper or....
% \geometry{margin=2in} % for example, change the margins to 2 inches all round
% \geometry{landscape} % set up the page for landscape
%   read geometry.pdf for detailed page layout information
\setlength{\parindent}{0pt}


% \usepackage[parfill]{parskip} % Activate to begin paragraphs with an empty line rather than an indent

%%% PACKAGES
\usepackage{booktabs} % for much better looking tables
\usepackage{array} % for better arrays (eg matrices) in maths
\usepackage{paralist} % very flexible & customisable lists (eg. enumerate/itemize, etc.)
\usepackage{verbatim} % adds environment for commenting out blocks of text & for better verbatim
\usepackage{subfig} % make it possible to include more than one captioned figure/table in a single float
\usepackage{amsmath} %amsmath is part of AMS-LATEX bundles
\usepackage{amssymb}%symble
\usepackage{amsfonts}%font
\usepackage{amsthm}%provide theorem package
\usepackage{graphicx}
\usepackage{listings}
% These packages are all incorporated in the memoir class to one degree or another...
\usepackage[retainorgcmds]{IEEEtrantools} %In order to use IEEEeqnarray Environment
\usepackage{graphicx} % support the \includegraphics command and options
%\usepackage{indentfirst}%

%%% HEADERS & FOOTERS
\usepackage{fancyhdr} % This should be set AFTER setting up the page geometry
\pagestyle{fancy} % options: empty , plain , fancy
\renewcommand{\headrulewidth}{0pt} % customise the layout...
\lhead{}\chead{}\rhead{}
\lfoot{}\cfoot{\thepage}\rfoot{}

%%% SECTION TITLE APPEARANCE
\usepackage{sectsty}
\allsectionsfont{\sffamily\mdseries\upshape} % (See the fntguide.pdf for font help)
% (This matches ConTeXt defaults)

%%% ToC (table of contents) APPEARANCE
\usepackage[nottoc,notlof,notlot]{tocbibind} % Put the bibliography in the ToC
\usepackage[titles,subfigure]{tocloft} % Alter the style of the Table of Contents
\renewcommand{\cftsecfont}{\rmfamily\mdseries\upshape}
\renewcommand{\cftsecpagefont}{\rmfamily\mdseries\upshape} % No bold!

%%%DEFINE UPRIGHT FONT MISSING FUNCTIONS????????????????????
\DeclareMathOperator{\argh}{argh}
\DeclareMathOperator*{\nut}{Nut}

%%%DEFINE NEW COMMANDS
\newcommand{\ud}{\,\mathrm{d}}

%%%DEFINE THEOREM
%\theoremstyle{definition} 
\theoremstyle{definition}\newtheorem{law}{Law}
\theoremstyle{plain}\newtheorem{jury}[law]{Jury}
\theoremstyle{remark}\newtheorem{juu}{Juu}
\theoremstyle{definition}\newtheorem{kuu}[law]{Kuu}
\theoremstyle{definition}\newtheorem{muu}{Muu}[section]
\theoremstyle{definition}\newtheorem{honoluu}{Honoluu}[section]
\theoremstyle{definition}\newtheorem{konoluu}[muu]{Konoluu}

%%% END Article customizations

%%% The "real" document content comes below...

\title{\textbf{ \begin{LARGE}Neural Network\end{LARGE}}\\ [0ex]\begin{Large} Homework 2 \end{Large} }
\author{Ning Ma, A50055399}
\date{} % Activate to display a given date or no date (if empty),
         % otherwise the current date is printed 

\begin{document}
\maketitle
\section{Perceptron}

{\bf 1.}
For 2-D, the decision boundary is 
\begin{equation}
w_1x_1 + w_2x_2 = \theta
\end{equation}
Assume ${x}$ is a point on the boundary. So, we have 
\begin{equation}
w^Tx= \theta
\end{equation}
Let $x^0$ be the point from which we want to compute the distance to the line. So, the distance is 
\begin{IEEEeqnarray*}{rCl}
\frac{w^\top(x - x^0)}{||w||_2} =  \frac{w^\top x- w^\top x^0}{||w||_2}
\end{IEEEeqnarray*}
So, the distance from the origin to the boundary is
\begin{equation}
\frac{w^\top x - w^\top0}{||w||_2} = \frac{w^\top x}{||w||_2}
\end{equation}

{\bf 2.}
\begin{enumerate}
\item[(a)] 
The learning rule is 
\begin{equation}
w_j(t + 1) = w_j(t) + \alpha (Teacher - Output)x_j
\end{equation}

\item[(b)] 
The perceptron learning goes as the following:
\begin{table}[htb]
\caption{perceptron learning for NAND}
\centering
\begin{tabular}{|c|c|c|c|c|c|c|c|}
\specialrule{.2em}{0em}{0.2em} 
$x_1$ & $x_2$ & $w_1$ & $w_2$ & $ Net $ & $Output$ & $Teacher$ & $ theta $\\
\hline
1 & 1 & 0 & 0 & 0 & 1 & 0 & 0\\
\hline
0 & 0 & -1 & -1 & 0 & 0 & 1 & 1\\
\hline
0 & 1 & -1 & -1 & -1 & 0 & 1 & 0\\
\hline
1 & 1 & -1 & 0 & -1 & 1 & 0 & -1\\
\hline
1 & 0 & -2 & -1 & -2 & 0 & 1 & 0\\
\hline
- & - & -1 & -1 & - & - & - & -1\\
\hline
\end{tabular}
\label{table:CPTsPolytree}
\end{table}\\
The final weights and threshold are $w_1$ = -1, $w_2$ = -1, $theta$ = -1.

\item[(c)]
This solution is not unique. For example, $w_1 = -2$, $w_2 = -2$, and $\theta = -2$ is another solution. Namely, the solution will  not change if you add same constant to all the parameters.

\item[(d)]
\begin{enumerate}
\item[i.]
See code file 'readProcessData.m' for details. After preprocessing, the data are unitless. If the data have different units, different data points may have very different numerical value but actually represent the same physical measurement.
 Secondly, if the original data is very large, this preprocessing may avoid overflowing and let the learning converge appropriately.
 \item[ii.]
 According to Figure~\ref{scatterPlot}, the class are linearly separable for most feature spaces.
 
 \item[iii.]
See code for more details. For the stopping criteria, I choose to stop when the iteration reaches 1000 or the error drops below 0.05, whichever occurs first.
\item[iv.]
The test error rate is just 3.33\%
\item[v.]
my learning rate is 0.05. When I slightly increase the learning rate, the perceptron learning will converge a little bit fast and vice versa. However, if my learning rate is too large, the learning procedure will not converge.
\end{enumerate}
\end{enumerate}


\section{Multilayer Perceptron}
{\bf (a)}
The cross entropy loss function for softmax regression is 
\begin{equation}
E = -\sum_{l = 0}^{K - 1}1_{\{label = l\}}\mathrm{log} \ y_l
\end{equation}
where $y_l = \frac{\mathrm{exp}(a_l)}{\sum_{m = 0}^{K - 1}\mathrm{exp}(a_m)}$

For the output layer, we have 
\begin{IEEEeqnarray}{rCl}
\delta_k & = & -\frac{\partial E}{\partial a_k} = -\sum_l \frac{\partial E}{\partial y_l}\frac{\partial y_l}{\partial a_k}\\
& = & -\sum_l-\frac{1_{\{label= l\}}}{y_l}(y_l\delta_{lk} - y_ly_k)\\
& = & \sum_l 1_{\{label = l\}}(\delta_{lk} - y_k)\\
& = & \sum_{l}\delta_{lk}1_{\{label = l\}} - y_k\sum_{l}1_{\{label = l\}}\\
& = & 1_{\{label = k\}} - y_k = t_k - y_k
\end{IEEEeqnarray}

For the hidden layer, $y_j = g(a_j)$, we have 
\begin{IEEEeqnarray}{rCl}
\delta_j & = & -\frac{\partial E}{\partial a_j} = -\sum_k \frac{\partial E}{\partial a_k}\frac{\partial a_k}{\partial a_j}\\
& = & \sum_k\delta_k\frac{\partial a_k}{\partial a_j} = \sum_k\delta_k\frac{\partial a_k}{\partial y_j}\frac{\partial{y_j}}{\partial{a_j}}\\
& = & \sum_k\delta_k\frac{\partial \sum_l w_{lk}y_l}{\partial y_j}y_j^{'} = \sum_k\delta_kw_{jk}y_j^{'}\\
& = & y_j^{'}\sum_k\delta_kw_{jk}
\end{IEEEeqnarray}
where $\delta_k$ has been computed from the output layer.

{\bf (b)}
For the output layer, we have 
\begin{IEEEeqnarray}{rCl}
w_{jk} & = & w_{jk} - \alpha\frac{\partial E}{\partial w_{jk}} = w_{jk} - \alpha\frac{\partial E}{\partial a_k}\frac{\partial a_k}{\partial w_{jk}}\\
& = & w_{jk}  + \alpha \delta_k \frac{\partial \sum_l w_{lk}y_l}{\partial w_{jk}}\\
& = & w_{jk} + \alpha \delta_k y_j
\end{IEEEeqnarray}

For the hidden layer, we have
\begin{IEEEeqnarray}{rCl}
w_{ij} & = & w_{ij} - \alpha\frac{\partial E}{\partial w_{ij}} = w_{ij} - \alpha\frac{\partial E}{\partial a_j}\frac{\partial a_j}{\partial w_{ij}}\\
& = & w_{ij}  + \alpha \delta_j \frac{\partial \sum_l w_{lj}x_l}{\partial w_{ij}}\\
& = & w_{ij} + \alpha \delta_j x_i
\end{IEEEeqnarray}
where we have already computed the $\delta_k$ and $\delta_j$ in part (a).

{\bf (c)}
For the output layer, since $ w_{jk} = w_{jk} + \alpha\delta_k y_j$, we have
\begin{IEEEeqnarray}{rCl}
W_{HO} = W_{HO} + \alpha y^{(j)} \otimes \delta^{(k)}
\end{IEEEeqnarray}
where $y^{(j)}$ is a column-vector output from hidden layer, $\delta^{(k)}$ is a column-vector $\delta$ from the output layer, and $\otimes$ is an outer product operator.\\

Similarly, for the hidden layer, since $w_{ij} =  w_{ij} + \alpha \delta_j x_i$, we have
\begin{IEEEeqnarray}{rCl}
W_{IH} = W_{IH} + \alpha x^{(i)} \otimes \delta^{(j)}
\end{IEEEeqnarray}
where $x^{(i)}$ is a column-vector input, $\delta^{(j)}$ is a column-vector $\delta$ from the  hidden layer, and $\otimes$ is an outer product operator.\\

Since $\delta_j = y_j^{'}\sum_k\delta_kw_{jk}$, we have 
\begin{IEEEeqnarray}{rCl}
\delta^{(j)} = (y^{'})^{(j)}\odot(W_{HO}\bullet\delta^{(k)})
\end{IEEEeqnarray}
where $\odot$ is an element-wise multiplication operator.
Thus, we have 
\begin{IEEEeqnarray}{rCl}
W_{IH} = W_{IH} + \alpha x^{(i)} \otimes \left ( (y^{'})^{(j)}\odot(W_{HO}\bullet\delta^{(k)}) \right )
\end{IEEEeqnarray}

{\bf (d)}
























{\bf 4.}
\begin{enumerate}
\item[(a)]
The result of the 10 2-way classification is in Table ~\ref{table:2-way}
\begin{table}[htb]
\caption{10 2-way classification}
\centering
\begin{tabular}{|c|c|c|c|c|c|c|c|c|c|}
\specialrule{.2em}{0em}{0.2em} 
$0-all$ & $1-all$ & $2-all$ & $3-all$ & $ 4-all $ & $5-all$ & $6-all$ & $ 7-all $ & $ 8-all $ & $ 9-all$\\
\hline
0.977 & 0.982 & 0.957 & 0.951 & 0.957 & 0.950 & 0.960 & 0.955 & 0.921 & 0.935\\
\hline
\end{tabular}
\label{table:2-way}
\end{table}
\item[(b)]
The overall test accuracy is 0.846.
\end{enumerate}

{\bf 5.}
\begin{enumerate}
\item[(a)]


\item[(b)]
The test accuracy on the test data is 0.860.

\item[(c)]
The test accuracy is a little bit hight than the one-vs-all logistic regression. It is because softmax regression can directly computes the probability of each class, and then vote for the one with highest probability. So, softmax regression will provide a little bit more accurate result. 
\end{enumerate}



\newpage
\section{Appendix}
The following is the  source code\\
\begin{lstlisting}
\end{lstlisting}

%P(X_1<X_2<X_3)=\int
%\\
%\textbf{5.3.2} Let $X$ and $Y$ be independent random variables, with $E(X)=1, E(Y)=2, Var(Y)=4.$\\
%\indent \setlength{\hangindent}{2em} (a) FInd $E(10X^2+8Y^2-XY+8X+5Y-1)$\\
%\indent \setlength{\hangindent}{2em} (b) Assuming all variables are normally distributed, find$ P(2X>3Y-5)$
%More text. 
\end{document}
